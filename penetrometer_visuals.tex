% Options for packages loaded elsewhere
\PassOptionsToPackage{unicode}{hyperref}
\PassOptionsToPackage{hyphens}{url}
\documentclass[
]{article}
\usepackage{xcolor}
\usepackage[margin=1in]{geometry}
\usepackage{amsmath,amssymb}
\setcounter{secnumdepth}{-\maxdimen} % remove section numbering
\usepackage{iftex}
\ifPDFTeX
  \usepackage[T1]{fontenc}
  \usepackage[utf8]{inputenc}
  \usepackage{textcomp} % provide euro and other symbols
\else % if luatex or xetex
  \usepackage{unicode-math} % this also loads fontspec
  \defaultfontfeatures{Scale=MatchLowercase}
  \defaultfontfeatures[\rmfamily]{Ligatures=TeX,Scale=1}
\fi
\usepackage{lmodern}
\ifPDFTeX\else
  % xetex/luatex font selection
\fi
% Use upquote if available, for straight quotes in verbatim environments
\IfFileExists{upquote.sty}{\usepackage{upquote}}{}
\IfFileExists{microtype.sty}{% use microtype if available
  \usepackage[]{microtype}
  \UseMicrotypeSet[protrusion]{basicmath} % disable protrusion for tt fonts
}{}
\makeatletter
\@ifundefined{KOMAClassName}{% if non-KOMA class
  \IfFileExists{parskip.sty}{%
    \usepackage{parskip}
  }{% else
    \setlength{\parindent}{0pt}
    \setlength{\parskip}{6pt plus 2pt minus 1pt}}
}{% if KOMA class
  \KOMAoptions{parskip=half}}
\makeatother
\usepackage{graphicx}
\makeatletter
\newsavebox\pandoc@box
\newcommand*\pandocbounded[1]{% scales image to fit in text height/width
  \sbox\pandoc@box{#1}%
  \Gscale@div\@tempa{\textheight}{\dimexpr\ht\pandoc@box+\dp\pandoc@box\relax}%
  \Gscale@div\@tempb{\linewidth}{\wd\pandoc@box}%
  \ifdim\@tempb\p@<\@tempa\p@\let\@tempa\@tempb\fi% select the smaller of both
  \ifdim\@tempa\p@<\p@\scalebox{\@tempa}{\usebox\pandoc@box}%
  \else\usebox{\pandoc@box}%
  \fi%
}
% Set default figure placement to htbp
\def\fps@figure{htbp}
\makeatother
\setlength{\emergencystretch}{3em} % prevent overfull lines
\providecommand{\tightlist}{%
  \setlength{\itemsep}{0pt}\setlength{\parskip}{0pt}}
\usepackage{bookmark}
\IfFileExists{xurl.sty}{\usepackage{xurl}}{} % add URL line breaks if available
\urlstyle{same}
\hypersetup{
  pdftitle={Penetrometer},
  pdfauthor={Rebecca Martinez},
  hidelinks,
  pdfcreator={LaTeX via pandoc}}

\title{Penetrometer}
\author{Rebecca Martinez}
\date{11-18-2025}

\begin{document}
\maketitle

\subsection{Methods and Measurments}\label{methods-and-measurments}

The drop cone method measures the depth of penetration resulting from a
cone of fixed mass being dropped from a standard height. The hammer-type
penetrometers use a slide hammer of fixed mass and drop height to apply
consistent kinetic energy with each blow. Either the number of blows
required to penetrate a specified depth, or the depth of penetration per
blow are measured in this method.

Soil penetration resistance \(R_s\) is defined as the work done by the
soil divided by the distance the penetrometer travels:

\[
R_s = \frac{W_s}{P_d}
\]

where \(R_s\) is the soil resistance (N), \(W_s\) is the work done by
the soil (J), and \(P_d\) is the distance the penetrometer travels (m).

The kinetic energy of a hammer of mass \(m = 4.58 \text{ kg}\) falling a
distance \(x = 0.45 \text{ m}\) is:

\[
v = \sqrt{v_0^2 + 2 a x} = \sqrt{0 + 2 (9.8)(0.45)} = 2.97 \text{ m/s}
\]

\[
KE = \frac{1}{2} m v^2 = \frac{1}{2} (4.58)(2.97)^2 = 20.20 \text{ J}
\]

Then the soil resistance is calculated as:

\[
R_s = \frac{W_s}{P_d} = \frac{KE}{P_d}
\]

\[
P_d = \frac{1}{blows/meter}
\]

So:

\[
R_s = \mult{KE}{P_d}
\]

\pandocbounded{\includegraphics[keepaspectratio]{penetrometer_visuals_files/figure-latex/blows vs depth-1.pdf}}

\pandocbounded{\includegraphics[keepaspectratio]{penetrometer_visuals_files/figure-latex/unnamed-chunk-2-1.pdf}}

\pandocbounded{\includegraphics[keepaspectratio]{penetrometer_visuals_files/figure-latex/unnamed-chunk-3-1.pdf}}

\pandocbounded{\includegraphics[keepaspectratio]{penetrometer_visuals_files/figure-latex/unnamed-chunk-4-1.pdf}}

\pandocbounded{\includegraphics[keepaspectratio]{penetrometer_visuals_files/figure-latex/unnamed-chunk-6-1.pdf}}

\end{document}
